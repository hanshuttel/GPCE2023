\documentclass[anonymous,review]{acmart}
\usepackage[utf8]{inputenc}
%\usepackage{lmodern}
\usepackage{uppunctlm}
\usepackage{amssymb, amsmath}
\usepackage{array, caption, booktabs}
\usepackage{stmaryrd}
\usepackage{graphicx} % Required for inserting images
\usepackage{semantic}
\usepackage{mathtools}
\usepackage{float}
\usepackage{cleveref}
\usepackage{xspace}
\usepackage{fancyvrb}
\usepackage{listings}
\usepackage{thmtools}

%\usepackage[
%backend=biber,
%style=numeric,
%sorting=ynt,
%maxcitenames=1,
%maxbibnames=99
%]{biblatex}
%\addbibresource{bibliography.bib}

\newcolumntype{L}{>{$}l<{$}} % left-oriented math column type
\newcolumntype{R}{>{$}r<{$}} % right-oriented math column type
\newcolumntype{C}{>{$}c<{$}} % center-oriented math column type

\newcommand{\lr}[1]{\langle #1 \rangle} % config macro: <#1>
\newcommand{\sub}[1]{\/\textsubscript{#1}} % subscript macro
\newcommand{\ltrans}[1]{\stackrel{#1}{\Rightarrow}} % labelled transition pil
\newcommand{\ltransn}[1]{\stackrel{#1}{\nRightarrow}} % labelled transition pil
\newcommand{\infeshort}[1]{\inference[{\normalsize(#1)}\;\;]} %formatting of \inference, when in a table side by side
\newcommand{\infelong}[1]{\inference[{\normalsize(#1)}\qquad]} %formatting of \inference, when on a single line
\newcommand{\define}{\stackrel{\text{def}}{=}} %def =
\newcommand{\encode}[1]{\llbracket #1 \rrbracket} %encoding brackets
\newcommand{\match}[1]{\{\mid #1 \mid\}} %encoding brackets
\newcommand{\child}{\textsf{child }} % 'child' stands out
\newcommand{\term}{$\lambda$-term\xspace} % lambda term
\newcommand{\OP}{\mathcal{O}} %family of operators
\newcommand{\OPc}{\hat{\mathcal{O}}} %cursorless family of operators
\newcommand{\OPcc}{\dot{\mathcal{O}}} %cursorless family of operators
\newcommand{\SORT}{\mathcal{S}} %set of sorts
\newcommand{\SORTc}{\hat{\mathcal{S}}} %set of cursorless sorts
\newcommand{\SORTcc}{\dot{\mathcal{S}}} %set of cursorless sorts
\newcommand{\VAR}{\mathcal{X}} %family of variables
\newcommand{\VARc}{\hat{\mathcal{X}}} %family of variables
\newcommand{\VARcc}{\dot{\mathcal{X}}} %family of variables
\newcommand{\AST}{\mathcal{A}[\VAR]} %family of asts
\newcommand{\ABT}[1]{\mathcal{B}[\VAR #1]} %family of abts
\newcommand{\ABTc}[1]{\hat{\mathcal{B}}[\VARc #1]} %family of cursorless abts
\newcommand{\ABTcc}[1]{\dot{\mathcal{B}}[\VARcc #1]} %family of cursorless abts
\newcommand{\alphac}{\rightarrow_\alpha} %alpha-conversion
\newcommand{\betar}{\rightarrow_\beta} %beta-reduction
\newcommand{\alphae}{\equiv_\alpha} %alpha-equivalence
\newcommand{\betae}{\equiv_\beta} %beta-equivalence
\newcommand{\stlc}{$\lambda_{->}$\xspace} %simply typed lambda calculus
\newcommand{\hole}[1]{\llparenthesis \ \rrparenthesis_#1} %hole, subscripted with a sort
\newcommand{\ctxhole}{[ \, \cdot \, ]} %context hole
\newcommand{\fix}{\text{fix}\xspace} %correct fix

\renewcommand{\vec}[1]{\overrightarrow{#1}}

\newcommand{\federeboks}[3]{
    \begin{figure}[H]
    \vspace{-3mm}
    \centering
    \fbox{%
        \footnotesize\renewcommand{\arraystretch}{1.25}
        \setlength{\tabcolsep} {4pt}
        \begin{tabular}{#1}
            \addlinespace[1.25ex]
            #2 \\
            \addlinespace[1.25ex]
        \end{tabular}
    }
    #3
    \vspace{-5mm}
    \end{figure}
}



%% \BibTeX command to typeset BibTeX logo in the docs
\AtBeginDocument{%
  \providecommand\BibTeX{{%
    Bib\TeX}}}

%% Rights management information.  This information is sent to you
%% when you complete the rights form.  These commands have SAMPLE
%% values in them; it is your responsibility as an author to replace
%% the commands and values with those provided to you when you
%% complete the rights form.
\setcopyright{acmcopyright}
\copyrightyear{2023}
\acmYear{2023}
\acmDOI{XXXXXXX.XXXXXXX}

\acmConference[GPCE 2023]{ }{July 07--07,
  2023}{Woodstock, NY}

\acmPrice{15.00}
\acmISBN{978-1-4503-XXXX-X/18/06}


%%
%% Submission ID.
%% Use this when submitting an article to a sponsored event. You'll
%% receive a unique submission ID from the organizers
%% of the event, and this ID should be used as the parameter to this command.
%%\acmSubmissionID{123-A56-BU3}

%%
%% For managing citations, it is recommended to use bibliography
%% files in BibTeX format.
%%
%% You can then either use BibTeX with the ACM-Reference-Format style,
%% or BibLaTeX with the acmnumeric or acmauthoryear sytles, that include
%% support for advanced citation of software artefact from the
%% biblatex-software package, also separately available on CTAN.
%%
%% Look at the sample-*-biblatex.tex files for templates showcasing
%% the biblatex styles.
%%

%%
%% The majority of ACM publications use numbered citations and
%% references.  The command \citestyle{authoryear} switches to the
%% "author year" style.
%%
%% If you are preparing content for an event
%% sponsored by ACM SIGGRAPH, you must use the "author year" style of
%% citations and references.
%% Uncommenting
%% the next command will enable that style.
%%\citestyle{acmauthoryear}


%%
%% end of the preamble, start of the body of the document source.
\begin{document}

%%
%% The "title" command has an optional parameter,
%% allowing the author to define a "short title" to be used in page headers.
\title{\emph{Response to reviewers:} A type-safe generalized editor calculus (Short Paper)}

%%
%% The "author" command and its associated commands are used to define
%% the authors and their affiliations.
%% Of note is the shared affiliation of the first two authors, and the
%% "authornote" and "authornotemark" commands
%% used to denote shared contribution to the research.

%%\orcid{1234-5678-9012}

%\author{Andreas Tor Mortensen}
%\email{atmo20@student.aau.dk}


%\author{Benjamin Bennetzen}
%\email{bbenne20@student.aau.dk}


%\author{Nikolaj Rossander Kristensen}
%\email{nrkr20@student.aau.dk}

%\author{Peter Buus Steffensen}
%%\authornotemark[1]
%\email{psteff19@student.aau.dk
%}

%\affiliation{%
%  \institution{Department of Computer Science at Aalborg University}
%  \streetaddress{Selma Lagerlöfsvej 300}
%  \city{Aalborg}
%  \country{Denmark}
%  \postcode{9220}
%}

%\author{Hans Huttel}
%\email{hans@cs.aau.dk}
%\affiliation{%
%  \institution{Department of Computer Science at Aalborg University}
%  \streetaddress{Selma Lagerlöfsvej 300}
 % \city{Aalborg}
 % \country{Denmark}
 % \postcode{9220}
%}



%%
%% By default, the full list of authors will be used in the page
%% headers. Often, this list is too long, and will overlap
%% other information printed in the page headers. This command allows
%% the author to define a more concise list
%% of authors' names for this purpose.
\renewcommand{\shortauthors}{AAU Bachelor}


%%
%% The code below is generated by the tool at http://dl.acm.org/ccs.cfm.
%% Please copy and paste the code instead of the example below.
%%
\begin{CCSXML}
<ccs2012>
   <concept>
       <concept_id>10011007.10011006.10011050.10011023</concept_id>
       <concept_desc>Software and its engineering~Specialized application languages</concept_desc>
       <concept_significance>500</concept_significance>
       </concept>
   <concept>
       <concept_id>10003752.10010124.10010131</concept_id>
       <concept_desc>Theory of computation~Program semantics</concept_desc>
       <concept_significance>500</concept_significance>
       </concept>
   <concept>
       <concept_id>10003752.10010124.10010131.10010134</concept_id>
       <concept_desc>Theory of computation~Operational semantics</concept_desc>
       <concept_significance>500</concept_significance>
       </concept>
 </ccs2012>
\end{CCSXML}

\ccsdesc[500]{Software and its engineering~Specialized application languages}
\ccsdesc[500]{Theory of computation~Program semantics}
\ccsdesc[500]{Theory of computation~Operational semantics}


%%
%% Keywords. The author(s) should pick words that accurately describe
%% the work being presented. Separate the keywords with commas.
\keywords{Syntax-directed editor, Generalisation, lambda calculus}
%% A "teaser" image appears between the author and affiliation
%% information and the body of the document, and typically spans the
%% page.
%\begin{teaserfigure}
 % \includegraphics[width=\textwidth]{Images/windmill.jpg}
 % \caption{Seattle Mariners at Spring Training, 2010.}
%  \Description{Enjoying the baseball game from the third-base
%  seats. Ichiro Suzuki preparing to bat.}
%  \label{fig:teaser}
%\end{teaserfigure}

%\received{20 February 2007}
%\received[revised]{12 March 2009}
%\received[accepted]{5 June 2009}

%%
%% This command processes the author and affiliation and title
%% information and builds the first part of the formatted document.
\maketitle

This response consists of

\begin{itemize}
\item Full responses to the comments and questions from the reviewers.
\item The full soundness proof, as requested by Reviewer 2.
\end{itemize}

\newcommand{\reviewer}[1]{\textsl{#1}}
\newcommand{\response}[1]{\paragraph{\textbf{Response:}} #1}

\section{Response}

\subsection{Review \#58A}

\begin{itemize}
\item \reviewer{L76: I see that Andersen et al. had to extend an
    existing editor calculus with let expressions, but I don't see how
    it is related to the previous sentence. Did they write the type
    safety proof from scratch?}
  \response{Some of the type safety proof had
  to be redone -- because of the presence of let-expressions
  in the underlying language and because of parametric
  polymorphism. The addition of a copy/paste primitive in the editor
  calculus was the cause of other revisions. }
\item \reviewer{L80: I could not figure out what "original work"
    refers to.}
  \response{This refers to the Cornell Program Synthesizer mentioned
    in lines 32-33.}
\item \reviewer{L91: I do not see any discussion of higher-order
    abstract syntax in the subsequent sections.}
\response{Indeed. We will rephrase this -- we simply take inspiration from
  this approach to representing abstract syntax.}
\item \reviewer{Example 2.1: Please explain what $s$ and $e$
    represent, and what the notation $e.s$ means. Also, I think the
    term "sort" is not used in the standard way.}
  \response{$s$ is a metavariable for "scoped expressions'' and $e$ is
  a metavariable for "simple expressions''. We will add this
  explanation. The notation $(e)s$ is the arity of the term constructor
  for scoped expressions that are simple expressions: it takes an $e$
  and creates an $s$.

Our presentation of abstract syntax and the terminology and notation
(includion the notions of sorts and
arities) closely follows that of Harper \cite{harper_foundations}, as
mentioned at the start of Section 2. We can elaborate on this approach
at that point in the text.}
\item \reviewer{Section 2.2: Please explain what each syntactic form
    represents.}
  \response{Editor expressions $E$ represent operations on abstract
    binding trees $a$. This has already been explained in
    Subsection 2.1.1. But indeed a short explanation of editor
    expressions in Section 2.1.1 is called for. It is:

    \begin{itemize}
    \item $\pi$ denote navigation actions that lets the cursor move
      down the abt ($\text{child} \; n$ and $\text{parent} $) and
      substitution actions that replace the subtree pointed to by the
      cursor by the syntactic operator $o$.
    \item Expressions $\pi.E$ denote performing $\pi$ and continuing
      as $E$. $\phi \Rightarrow E_1|E_2$ is a conditional expression whose
      continuation depends on the condition $\phi$. $E_1 \ggg E_2$ is
      sequential composition. $\text{rec}\;x.E$ is a recursive
      expression; occurrences of $x$ in $E$ are recursive
      calls. $\text{nil}$ is the empty editor expression.
    \end{itemize}

    We will add this explanation to Section 2.1.1.}
  
\item \reviewer{L174: It is not very clear what the authors mean by
    "encapsulated by the cursor".}
  \response{The cursor points to the root a subtree of the abt being
    edited, so the abt $a$ can be written as a cursor context $C[a_1]$
    whenever the cursor is at the root of the subtree $a_1$. See also
    the end of Section 2.1.2.}
\end{itemize}

\bibliographystyle{ACM-Reference-Format}
\bibliography{bibliography}


\end{document}
\endinput

%%% Local Variables:
%%% mode: latex
%%% TeX-master: t
%%% End:
